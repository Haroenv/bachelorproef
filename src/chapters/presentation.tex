% !TEX root = ../main.tex

\section{Presentation of the company} % (fold)
\label{sec:presentation}

\subsection{The company} % (fold)
\label{sub:company}

Algolia\cite{algolia-home} is a company founded in Paris, but now has its headquarters in San Fransisco, US, although most of the development happens in Paris. The goal is to help people discover content they want to find, which is achieved by a SaaS search offering.

People upload their data onto Algolia’s infrastructure, where the data is split up into n-grams3. This results in a nested datastructure that’s very fast to search in and has features like typo-tolerance, faceting etc. at the cost of a slightly higher indexing time (a few seconds to a minute).

The story doesn’t end there, because developer experience (DX) is held in a very high standard at Algolia. There are API clients for a lot of languages5. In JavaScript and mobile the offering goes further than that, by offering what’s called instantsearch6789. These are a set of libraries orchestrating all parts of a good search experience.

\begin{figure}[H]
  \label{figure:company-logo}
  \centering
  \includegraphics[width=0.5\textwidth]{../assets/algolia-logo-light.pdf}
  \caption{The logo of Algolia \cite{algolia-press}}
\end{figure}

\subsubsection{Work environment}
\label{sub:work-environment}

Citeer bronnen van websites \cite{voorbeeld-ref} en boeken \cite{boek-ref}.

\lipsum[1]

