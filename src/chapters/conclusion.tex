% !TEX root = ../main.tex

\chapter{Conclusion} % (fold)
\label{chp:conclusion}

\section{Technical reflection}

During this internship at the Paris office of Algolia the practical research on how to reimagine InstantSearch.js as a framework-agnostic library was executed.

Getting to know the company and their core values was important in this culture-oriented startup. Working in an environment on a day to day base with \textbf{grit} proves to be stimulating and gets the best out of everyone involved. As an intern getting \textbf{trust} and feeling a valuable asset for the company and community was very motivating. Taking \textbf{care} for co-workers and clients, but also feeling \textbf{cared} for made even the more challenging tasks achievable. Getting honest feedback from others with \textbf{candor} gave the research direction and kept in on track. Giving feedback to others without giving the impression as an intern to loose one's \textbf{humility} is a fine balance to find.

Exploring the existing and previous libraries as well as the open source package manager Yarn and experiencing the customer support as done by Algolia was the first step in this research. This was the foundation for further development.

Making a complete search experience for a website that has a big user base, and being able to increase the usage of search almost twentyfold is very fulfilling to be a part of. Having insight in a website that is used to look up your own dependencies on a day-to-day basis is very rewarding, since it's improving not only one's own experience, but those of thousands of developers.

Interacting with clients as a newcomer is scary in the beginning. Once overcoming this fear it improved the insight in the existing libraries and boosts one's confidence when improvements in parts of the logic are appreciated and applauded.

Reimagining InstantSearch as a new set of paradigms proved to be work that could lovely be solved using the connectors concept. Figuring out a way to take this knowledge to a lower level than InstantSearch, and actually suggesting improvements to the JavaScript Helper is an ongoing work, that will prove to stay part of the authors responsibilities, has a very big impact.

\section{Personal reflection}

This internship and the associated practical study proved to be a very meaningful experience: not only improving programming skills and learning to work in a community, but also stimulated personal growth. 

The existing programming skills improved not only during the experimental research with day-to-day discoveries of new things. Reflecting on the found solutions in one-to-one meetings where Sepehr Fakour was a critical friend and asked  deep questions on the found results and planed actions. Deciding to focus on learning as much as possible on the solutions that already  are in place to solve problems Algolia has opened my eyes in understanding paradigms that when trying to understand them earlier, went over my head.

Working in a community is meant twofold. Algolia is focused on its  core values, which causes one to take part in a lot of different environments, far from the task that is originally assigned.

Working on a product that actually is used by a lot of people is very humbling, realising  that every action taken has an impact. Users are aware of an accidentally broken build of a library, but fixing an existing bug will be appreciated.

Contributing on the search experience and detail page of Yarn has been, and is one of the  biggest single contributions I’ve done to a project that didn't start out as mine. It’s very interesting to see people’s opinion on your work, as well as finding how to fit in what you’re making with  what already exists.

Doing this internship in a city where I’ve never been a long time before, as well as being independent for a significant period of time has been another great life experience for emotional development, rather than cognitive. Doing this internship has convinced me that I’m capable of living in Paris, and making a contribution to Algolia and the community.
