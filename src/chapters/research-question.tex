% !TEX root = ../main.tex

\chapter{Research question} % (fold)
\label{chp:research_question}
\section{Question}
\label{sec:question}

Reimagining InstantSearch.js as a framework-agnostic library is the main goal of the research in the internship and this bachelor thesis.

InstantSearch.js is a JavaScript library that simplifies making search user interfaces with Algolia. It provides a version for vanilla JavaScript environments, which uses React fundamentals behind the screens, as well as a version for React and soon for Vue.

When making these new versions, a feeling of duplication comes up. Avoiding this feeling is imperative, since there is a loss of consistency and ease of development. This is a major challenge and is topic for practical research during the internship and in the bachelor thesis.

\section{Operational goals}
\label{sec:operational_goals}

\subsection{Exploration}

Experimental research is a very important part of this work. To be able to propose solutions for the creation of libraries, a lot of insight in the goals of the libraries is needed. This is translated in three major points. 

Firstly there is need for deep contribution in the existing libraries. Only by contributing to the development of the different flavours of InstantSearch, insights can be found that are the slight differences between each product.

Secondly customer support is an important tool to get a feeling of what the users actually want to use the libraries for. Often what you, as a library author, expect a library to be used for, might be too constrained for advanced users.

Finally the exploration of the libraries is also done by using them in a well-frequented production website. This website is the search and detail experience for Yarn\cite{yarn-site}~. This website is converted from using InstantSearch.js to use React InstantSearch, as well as using the Algolia JavaScript client for specific parts.

By working on a website that has many different approaches on loading data from Algolia indices and other sources, an understanding of a wide spectrum of the ecosystem is acquired.

\subsection{Experimentation}

After this insight a solution will be proposed, which takes as much good ideas from InstantSearch.js, Vue InstantSearch and React InstantSearch. This proposed solution will then be validated by integrating it as in a single widget as an example.

Chapter \ref{chp:execution} will be about this experimental new API, which has as goal to be still as low-level as the JavaScript Helper, as well as allowing for simplicity that fits in with a component-based model.
