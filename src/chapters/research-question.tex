% !TEX root = ../main.tex

\chapter{Research question} % (fold)
\label{chp:research_question}
\section{Question}
\label{sec:question}

Reimagining InstantSearch.js as a framework-agnostic library. InstantSearch.js is a JavaScript library that simplifies making search user interfaces with Algolia. 

It provides a version for vanilla JavaScript environments, which uses React fundamentals behind the screens, as well as a version for React and soon for Vue.

When making these new versions, a feeling of duplication comes up. This is not the most productive environment, since there is a loss of consistency, and ease of development.

\section{Operational goals}
\label{sec:operational_goals}

Firstly experimentation is seen as a very important part of this work. To be able to propose solutions, a lot of insight in the goals of the libraries is needed. 

This is translated in three major points. Firstly there is need for deep contribution in the existing libraries. Secondly customer support, to get a feeling of what the users actually want to use the libraries for. Often what you, as a library author, expect a library to be used for, might be too constrained for advanced users.

Finally the exploration of the libraries is also done by using them in a well-frequented production website. This website is the search and detail experience for Yarn\cite{yarn-site}~. This website is converted from using InstantSearch.js to use React InstantSearch, as well as using the Algolia JavaScript client for specific parts.

By working on a website that has many different approaches on loading data from Algolia indices and other sources, an understanding of a wide spectrum of the ecosystem is acquired.

After this insight a solution will be proposed, which takes as much good ideas from InstantSearch.js, Vue InstantSearch and React InstantSearch. This proposed solution will then be validated by integrating it as in a single widget as an example.
